\documentclass{beamer}
\mode<presentation>
{
  \usetheme{Warsaw}      % or try Darmstadt, Madrid, Warsaw, ...
  \usecolortheme{beaver} % or try albatross, beaver, crane, ...
  \usefonttheme{default}  % or try serif, structurebold, ...
  \setbeamertemplate{navigation symbols}{}
  \setbeamertemplate{caption}[numbered]
  \setbeamertemplate{footline}[frame number]
}


\usepackage[english]{babel}
\usepackage{chronology}
\usepackage[utf8]{inputenc}
\usepackage{pgfpages}
\usepackage{algorithm}
\usepackage{color}
\usepackage{verbatim}
\usepackage[noend]{algorithmic}
\usepackage{hyperref}

\usepackage{calc}
\usepackage{ifthen}
\usepackage{tikz}

\title {Usability \& Skeuomorphism}
\subtitle {INFO-F-501 Information technology in society}
\author{Thomas~Chapeaux  \and Bernard~Mayeur \and Xavier~Bodart }
\institute[shortinst]{Université Libre de Bruxelles \\ Belgium}
\date{December 2013}

\begin{document}

\maketitle{}

\begin{frame}{Outline}
    \tableofcontents
\end{frame}

\section{Context}

\subsection{Design principles}

\begin{frame}{Outline}
    \tableofcontents[currentsection]
\end{frame}



\begin{frame}{Perceived affordance}
	\begin{block}{Perceived affordance}
	Perceived affordance is the quality of an object that suggests how it might be used.
	\end{block}
	\begin{figure}[ht]
	\includegraphics[scale=0.3]{switches.png}
	\end{figure}
\end{frame}

\begin{frame}{Natural mapping}
	\begin{block}{Natural mapping}
	The proper and natural arrangements for the relations between controls and their 	movements to the outcome from such action into the world.
	%The real function of natural mappings is to reduce the need for any information 	from a user’s memory to perform a task.
	\end{block}
	\begin{figure}[ht]
	\includegraphics[scale=0.3]{stove_natural.png}
	\end{figure}
	\end{frame}

\begin{frame}{Feedback}
	\begin{block}{Feedback}
	Sending back to the user information about what action has actually been done, what result has been accomplished.
	\end{block}
	\begin{figure}[ht]
	\includegraphics[scale=0.3]{retro-action-speaking.jpg}
	\end{figure}
\end{frame}

\begin{frame}{Logical constraints}
    \begin{itemize}
    		\item User-friendliness
    		\item Call to the human good-sense
            \item Can be induced by natural mapping
    		\item Car windows switches
    \end{itemize}
     \begin{figure}
             \includegraphics[scale=0.25]{window.jpg}
             \end{figure}
\end{frame}

\begin{frame}{Physical constraints}
    %Left switch for the left lights in a room%
    		 \begin{itemize}
    		 \item Strongest constraints
    		 \item Constrained by physical laws
    		 \item Toolbox
    		 \end{itemize}
             \begin{figure}
             \includegraphics[scale=0.1]{toolbox.jpg}
             \end{figure}

	%Flat key in a flat keyhole%
\end{frame}

\begin{frame}{Cultural constraints}
    \begin{itemize}
		\item Long term learning
    		\item Known situations that can be replicated
    		\item Not universal
    \end{itemize}
    \begin{figure}[ht]
    \includegraphics[scale=0.1]{japan-traffic-light.jpg}
    \end{figure}
\end{frame}

\begin{frame}{Semantic constraints}
    \begin{itemize}
    \item Impossible at first try
    \item Short term learning
    \end{itemize}
    \begin{figure}
    \includegraphics[scale=0.4]{tie_your_shoes.jpg}
    \end{figure}
\end{frame}

\subsection{History of IT design}

\begin{frame}{Outline}
    \tableofcontents[currentsection]
\end{frame}

\begin{frame}{Evolution in IT:Timeline}

\begin{enumerate}
\item \textbf{1920} Concept of vocal interface: Radio rex
\item \textbf{1960$\rightarrow$1965} Command line interface invention
\item \textbf{1963} Invention of the mouse
\item \textbf{1965$\rightarrow$1985} Massive utilization of CLI
\item \textbf{1970} Neural activity study on monkeys
\item \textbf{1973} First utilisation of a touchscreen (CERN)
\item \textbf{1980 $\rightarrow$ now} Utilization of graphical interface
\item \textbf{early 1980's} Standardization of the mouse
\item \textbf{1983} First commercialized touchscreen computer (by HP)
\item \textbf{1990's $\rightarrow$ now} Standardization of touch interface(e.g PDA)
\item \textbf{end of 1990's $\rightarrow$ now} Standardization of vocal interface(e.g Dragon naturally speaking)
\item \textbf{2000} First proving results of neural interface
\end{enumerate}
\end{frame}

\begin{frame}{Evolution in IT: Command line interface}

\begin{itemize}
\item Command interpreter
\item Lack of affordance
\end{itemize}
	\begin{figure}
			  \begin{minipage}{5cm}
				  \includegraphics[width=8cm]{terminal.png}
			  \end{minipage}

		\end{figure}
\end{frame}
\begin{frame}{Evolution in IT: Mouse \& Graphical interface}
\begin{itemize}
\item Used first to be able to navigate into the NLS
\item Coupled with graphical itnerface
\item Invented by Xerox and standardized by Apple
\item Gain of affordance + natural mapping + feedback
\end{itemize}

\begin{figure}
\raggedleft
	   \begin{minipage}{5cm}
				  \includegraphics[width=3cm]{mice.jpg}
			  \end{minipage}
			  	   \begin{minipage}{5cm}
				  \includegraphics[width=2cm]{doug.jpg}
			  \end{minipage}
\end{figure}
\end{frame}

\begin{frame}{Evolution in IT: Touch interface}
\begin{itemize}
\item Multiple technologies
\item Natural affordance $\rightarrow$ Mouse $=$ Finger
\item Instant feedback
\item Complex operations using multiple finger
\end{itemize}
\begin{figure}[ht]
\includegraphics[scale=0.3]{multitouch_gestures_trackpad_2.jpg}
\end{figure}
\end{frame}
\begin{frame}{Evolution in IT: Vocal interface}
\begin{itemize}
\item From isolated words to spontaneous speech
\item 2 main evaluation criteria
\begin{itemize}
	\item Speed
	\item Accuracy
\end{itemize}
\item Grant maximum usability if nice trained.
\end{itemize}
\begin{figure}[ht]
\begin{minipage}[b]{0.30\linewidth}
\centering
\includegraphics[width=\textwidth]{radio-rex.jpg}


\end{minipage}
\hspace{0.15cm}
\begin{minipage}[b]{0.40\linewidth}
\centering
\includegraphics[width=\textwidth]{recognition-process.png}


\end{minipage}
\end{figure}
\end{frame}

\begin{frame}{Evolution in IT : Neuronal interface}
\begin{itemize}
\item Extremely expensive (Still in research)
\item Invasive or Non-invasive interface
\item Matt Nagle case
\end{itemize}
\begin{figure}[ht]
\begin{minipage}[b]{0.30\linewidth}
\centering
\includegraphics[width=\textwidth]{matt-nagle.jpg}


\end{minipage}
\hspace{0.15cm}
\begin{minipage}[b]{0.40\linewidth}
\centering
\includegraphics[width=\textwidth]{brain-chip.jpg}
\end{minipage}
\end{figure}
\end{frame}

\subsection{Touchscreen interface: Flat vs. Skeuomorphism}

\begin{frame}{Outline}
    \tableofcontents[currentsection]
\end{frame}


	\begin{frame}{Skeuomorphism}
	\begin{block}{Skeuomorph}
	An object or feature which imitates the design of a similar artefact made from another material.
	\end{block}
	\begin{itemize}
	\item Skeuos: $\sigma \kappa \varepsilon \upsilon {\rm o} \zeta$ = Tool(or container)
	\item Morphê: $\mu {\rm o} \rho \varphi \eta$ = shape
	\item Applied to physical and computer/mobile interfaces
	\end{itemize}
        \begin{figure}
    \includegraphics[scale=0.25]{idial.jpg}
    \end{figure}


	\end{frame}
	\begin{frame}{Flat design}
	\begin{block}{Flat design}
	Flat design is a minimalistic design approach that emphasizes usability. It features clean, open, crisp edges, bright colours and two-dimensional/flat illustrations.
	\end{block}
    \begin{figure}
    \includegraphics[scale=0.35]{dial-flat.png}
    \end{figure}

	\end{frame}

    \begin{frame}
      \begin{columns}[t]
          \begin{column}{5cm}
           \Large\textbf{Skeuomorph\\}
           \normalsize
             	\textbf{Pros}
                  \begin{itemize}
                  \item Affordance gained by design
                  \item Attractive design for new users
                  \item User-friendly
                  \end{itemize}
                  \textbf{Cons}
                  \begin{itemize}
                  \item Can introduce delays
                  \item Can be felt as not serious for working applications
                  \item Time spent for the design and not the application
                  \item Does not always represent the trend of a given object
                  \end{itemize}
            \end{column}
            \begin{column}{5cm}
          \LARGE\textbf{Flat-design\\}
          \normalsize
       \textbf{Pros}
          \begin{itemize}
          \item As simple as possible
          \item Load times and "stylesheets" tends to be smaller
          \item Enhancing of the usability
          \end{itemize}
          \textbf{Cons}
          \begin{itemize}
              \item Relations between objects can be confusing (color and shapes can help): loss of affordance
          \item Limited in design (hard to make something new and elegant)
          \end{itemize}
       \end{column}


\end{columns}

    \end{frame}

\section{Experimentation}

\begin{frame}{Outline}

	\tableofcontents[currentsection]

\end{frame}

\subsection{Definition of the experiment}
\begin{frame}
We compared two different calculator apps on iPad.

\begin{columns}[c]
	\column{.5\textwidth}
    \begin{center}
    \includegraphics[height=.75\textwidth]{soulver_screenshot.png}

    Soulver - a flat calculator
    \end{center}
    \column{.5\textwidth}
	\begin{center}
    \includegraphics[height=.75\textwidth]{soulver_screenshot.png}

    m48+ Emulator
    \end{center}
\end{columns}

A group of $n$ subjects were asked to perform daily tasks with each of the application.

\end{frame}

\subsection{Results}

\begin{frame}{Results}


\end{frame}

\section{Previous results}
\subsection{Flat VS Skeuomorph icons (June 2013)}

\begin{frame}{Outline}
    \tableofcontents[currentsection]
\end{frame}

\begin{frame}{Flat VS Skeuomorph icons}
	\begin{figure}
	\centering
	\includegraphics[scale=0.5]{flat.png}
	\end{figure}
    \begin{center}
      \begin{tabular}{|l|c|c|c|c|}
        \hline
        &Windows 8 & IOS7 & IOS6 & Google\\
        \hline
        Video App 	& 19\%	& 22\%	& 21\%	& 38\% \\
        Calendar App& 10\%	& 44\%	& 23\%	& 23\% \\
        Video App	& 24\%	& 22\%	& 30\%	& 24\% \\
        \hline
        Average		& 17.6\%& 29.3\%& 24.6\%& 28.3\% \\
        \hline
      \end{tabular}
    \end{center}
    	\begin{flushright}\tiny\url{www.grapheine/divers/flat-design-vs-skeumorphisme}\normalsize\end{flushright}
\end{frame}

\subsection{Windows survey (Sept 2012)}
\begin{frame}{Windows survey about new users of windows 8}

    \small
    \begin{center}
      \begin{tabular}{|l|c|c|c|c|}
        \hline
        & Windows 8 & Windows 7 & Windows XP & Others\\
        \hline
        Already used windows & 26\%	& 75\%	& 58\%	& 17\% \\
        Prefered windows	 & 25\%	& 53\%	& 20\%	& 2\% \\
        \hline
      \end{tabular}
    \end{center}
    \normalsize
    \begin{columns}[c]
      \column{.5\textwidth}
      \begin{figure}
        \centering
        \includegraphics[scale=0.25]{windows8UI.jpg}
      \end{figure}
      \column{.5\textwidth}
      \begin{figure}
        \centering
        \includegraphics[scale=0.25]{windows8Phone.png}

        \centering
        \includegraphics[scale=0.25]{windows8Tablet.png}
      \end{figure}

	\end{columns}
	\begin{flushright}\tiny\url{www.forumswindows8.com/general-discussion/windows-8-survey-half-prefer-windows-7-a-7853.htm}\normalsize\end{flushright}
\end{frame}



\subsection{Japanese study about concrete and flat design (March 2013)}
\begin{frame}{Japanese study about concrete and flat design}
	\begin{itemize}
    \item Vast choice of app, first impression is important. Design of the icon should be well choosen\\
    \begin{figure}
        \centering
        \includegraphics[scale=0.5]{UI.png}
      \end{figure}
    \item Icons analysed by 8 designers with one questionnaire, and by 42 buyers of apps with another. Tried to find best correlation between particularities to make a map
    \end{itemize}
	\begin{flushright}\tiny\url{http://design-cu.jp/iasdr2013/papers/1811-1b.pdf}\normalsize\end{flushright}
\end{frame}

\begin{frame}{Japanese study about concrete and flat design (2)}
    \begin{figure}
        \centering
        \includegraphics[scale=0.45]{caract.png}
      \end{figure}
	\begin{flushright}\tiny\url{http://design-cu.jp/iasdr2013/papers/1811-1b.pdf}\normalsize\end{flushright}
\end{frame}

\begin{frame}{Japanese study about concrete and flat design (3)}
	\begin{columns}[c]
      \column{.6\textwidth}
      \begin{figure}
        \centering
        \includegraphics[scale=0.45]{map.png}
      \end{figure}
      \column{.4\textwidth}
      \begin{itemize}
        \item Preference between 16 icons
        \item 77 participants
        \item 77\% smartphones
        \item 23\% featurephones
      \end{itemize}
	\end{columns}
	\begin{flushright}\tiny\url{http://design-cu.jp/iasdr2013/papers/1811-1b.pdf}\normalsize\end{flushright}
\end{frame}

\begin{frame}{Japanese study about concrete and flat design (4)}
	\begin{figure}
    	\centering
        \includegraphics[scale=0.6]{finalStat.png}
    \end{figure}
    \begin{itemize}
        \item 75\% prefer concrete icons
		\item Icon music is different... Old item not used anymore?
    \end{itemize}
	\begin{flushright}\tiny\url{http://design-cu.jp/iasdr2013/papers/1811-1b.pdf}\normalsize\end{flushright}
\end{frame}

%\bibliographystyle{amsalpha}
%\bibliography{dit-paper}

\end{document}
