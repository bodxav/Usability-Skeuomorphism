\documentclass{beamer}
\mode<presentation>
{
  \usetheme{Warsaw}      % or try Darmstadt, Madrid, Warsaw, ...
  \usecolortheme{beaver} % or try albatross, beaver, crane, ...
  \usefonttheme{default}  % or try serif, structurebold, ...
  \setbeamertemplate{navigation symbols}{}
  \setbeamertemplate{caption}[numbered]
  \setbeamertemplate{footline}[frame number]
}


\usepackage[english]{babel}
\usepackage{chronology}
\usepackage[utf8]{inputenc}
\usepackage{pgfpages}
\usepackage{algorithm}
\usepackage{color}
\usepackage{verbatim}
\usepackage[noend]{algorithmic}
\usepackage{hyperref}

\usepackage{calc}
\usepackage{ifthen}
\usepackage{tikz}

\title{Usability of touch-based devices \\ Skeuomorph and Flat design}
\subtitle {INFO-F-501 Information technology in society}
\author{Thomas~Chapeaux  \and Bernard~Mayeur \and Xavier~Bodart }
\institute[shortinst]{Université Libre de Bruxelles \\ Belgium}
\date{January 2014}

\begin{document}

\maketitle{}

\begin{frame}{Outline}
    \tableofcontents
\end{frame}

\section{Context}

\subsection{Design principles}

\begin{frame}{Outline}
    \tableofcontents[currentsection]
\end{frame}



\begin{frame}{Perceived affordance}
	\begin{block}{Perceived affordance}
	Perceived affordance is the quality of an object that suggests how it might be used.
	\end{block}
	\begin{figure}[ht]
	\includegraphics[scale=0.3]{switches.png}
	\end{figure}
\end{frame}

\begin{frame}{Natural mapping}
	\begin{block}{Natural mapping}
	The proper and natural arrangements for the relations between controls and their 	movements to the outcome from such action into the world.
	%The real function of natural mappings is to reduce the need for any information 	from a user’s memory to perform a task.
	\end{block}
	\begin{figure}[ht]
	\includegraphics[scale=0.3]{stove_natural.png}
	\end{figure}
	\end{frame}

\begin{frame}{Feedback}
	\begin{block}{Feedback}
	Sending back to the user information about what action has actually been done, what result has been accomplished.
	\end{block}
	\begin{figure}[ht]
	\includegraphics[scale=0.3]{retro-action-speaking.jpg}
	\end{figure}
\end{frame}

\begin{frame}{Physical constraints}
    %Left switch for the left lights in a room%
    		 \begin{itemize}
    		 \item Strongest constraints
    		 \item Constrained by physical laws
    		 \item Toolbox
    		 \end{itemize}
             \begin{figure}
             \includegraphics[scale=0.1]{toolbox.jpg}
             \end{figure}

	%Flat key in a flat keyhole%
\end{frame}

\begin{frame}{Logical constraints}
    \begin{itemize}
    		\item User-friendliness
    		\item Call to the human good-sense $\Rightarrow$ Elimination
    		\item First assumption  needs to be the good one.
    		\item Frying pan handle
    \end{itemize}

        \begin{figure}[h]
        \centering
        \includegraphics[width=0.45\textwidth]{frying-pan-2handles.jpg}
        \includegraphics[width=0.43\textwidth]{frying-pan-1handle.jpg}
        \end{figure}
\end{frame}



\begin{frame}{Cultural constraints}
    \begin{itemize}
		\item Long term learning
    		\item Known situations that can be replicated
    		\item Not universal
    \end{itemize}
    \begin{figure}[ht]
    \includegraphics[scale=0.1]{japan-traffic-light.jpg}
    \end{figure}
\end{frame}

\begin{frame}{Semantic constraints}
    \begin{itemize}
    \item Impossible at first try
    \item Weakest constraints
    \item Short term learning
    \end{itemize}
    \begin{figure}
    \includegraphics[scale=0.4]{tie_your_shoes.jpg}
    \end{figure}
\end{frame}

\subsection{History of IT design}

\begin{frame}{Outline}
    \tableofcontents[currentsection, currentsubsection]
\end{frame}

\begin{frame}{Evolution in IT:Timeline}
\begin{figure}
\centering
\includegraphics[scale=0.3]{timeline-interface_2.png}
\end{figure}
\end{frame}

\begin{frame}{Evolution in IT: Command Line Interface}

\begin{itemize}
\item First attempt at design
\item Keyboard: labeled buttons (logical constraint)
\item Basic feedback
\item Lack of affordance
\end{itemize}
	\begin{figure}
			  \begin{minipage}{5cm}
				  \includegraphics[width=8cm]{terminal.png}
			  \end{minipage}

		\end{figure}
\end{frame}

\begin{frame}{Evolution in IT: Mouse and Graphical User Interface}
\begin{itemize}
\item Invented by Xerox and standardized by Apple
\item Simpler conceptual model (the user can ``move'' inside the screen) thanks to natural mapping
\item Better affordance and feedback
\end{itemize}

\begin{figure}
\raggedleft
	   \begin{minipage}{5cm}
				  \includegraphics[width=3cm]{mice.jpg}
			  \end{minipage}
			  	   \begin{minipage}{5cm}
				  \includegraphics[width=2cm]{doug.jpg}
			  \end{minipage}
\end{figure}
\end{frame}

\begin{frame}{Evolution in IT: Touch interface}
\begin{itemize}
\item Natural affordance: direct interaction with the screen
\item Even simpler conceptual model
\item Complex operations using multiple finger
\item But...
    \begin{itemize}
        \item Less precise than a mouse
        \item Artificial feel
        \item Lack of physical feedback
    \end{itemize}
\end{itemize}
\begin{figure}[ht]
\includegraphics[scale=0.3]{multitouch_gestures_trackpad_2.jpg}
\end{figure}
\end{frame}

\begin{frame}{Evolution in IT: Vocal interface}
\begin{itemize}
\item Old idea, recently made possible
\item Closely linked to technology
\item Conceptual model: ask the computer anything
\item In practice: not usable yet
\end{itemize}
\begin{figure}[ht]
\begin{minipage}[b]{0.30\linewidth}
\centering
\includegraphics[width=\textwidth]{radio-rex.jpg}


\end{minipage}
\hspace{0.15cm}
\begin{minipage}[b]{0.40\linewidth}
\centering
\includegraphics[width=\textwidth]{recognition-process.png}

\end{minipage}
\end{figure}
\end{frame}

\begin{frame}{Evolution in IT : Neuronal interface}
\begin{itemize}
\item Expensive (Still in research)
\item Invasive or Non-invasive interface
\item Matt Nagle case
\end{itemize}

~\\

\begin{center}
    \includegraphics[width=0.4\textwidth]{matt-nagle.jpg} ~
    \includegraphics[width=0.4\textwidth]{brain-chip.jpg}
\end{center}

\end{frame}

\subsection{Touchscreen interface: Flat vs. Skeuomorphism}

\begin{frame}{Outline}
    \tableofcontents[currentsection, currentsubsection]
\end{frame}

\begin{frame}{Skeuomorphism}
\begin{block}{Skeuomorph}
An object or feature which imitates the design of a similar artefact made from another material.
\end{block}
\begin{itemize}
\item Skeuos: $\sigma \kappa \varepsilon \upsilon {\rm o} \zeta$ = Tool, Container
\item Morphê: $\mu {\rm o} \rho \varphi \eta$ = Shape
\item Applied to physical and computer/mobile interfaces
\end{itemize}
    \begin{figure}
\includegraphics[scale=0.25]{idial.jpg}
\end{figure}


\end{frame}
\begin{frame}{Flat design}
\begin{block}{Flat design}
Flat design is a minimalistic design approach that emphasizes usability. It features clean, open, crisp edges, bright colours and two-dimensional/flat illustrations.
\end{block}
\begin{figure}
\includegraphics[scale=0.35]{dial-flat.png}
\end{figure}

\end{frame}

\begin{frame}{Comparison}
    \begin{columns}[c]
        \column{.5\textwidth}
            Skeuomorphism
            \begin{itemize}
                \item Intuitive conceptual model
                \item Sense of familiarity
                \item Limited by previous design
            \end{itemize}
        \column{.5\textwidth}
            Flat
            \begin{itemize}
                \item Clean look
                \item Spatially efficient
                \item (Far) less intuitive
            \end{itemize}
    \end{columns}
    \begin{columns}[c]
                \column{.5\textwidth}
                    \begin{center}
                    \includegraphics[width=0.7\textwidth]{idial.jpg}
                    \end{center}
                \column{.5\textwidth}
                    \begin{center}
                    \includegraphics[width=0.4\textwidth]{dial-flat.png}
                    \end{center}
    \end{columns}
\end{frame}

\section{Experimentation}

\begin{frame}{Outline}

	\tableofcontents[currentsection]

\end{frame}

\subsection{Definition of the experiment}
\begin{frame}{Experiment - definition}

\begin{itemize}
    \item We compared two different calculator apps on iPad.
    \item After a short demo of each application, a group of 10 subjects were asked to perform daily tasks with them.
\end{itemize}

\begin{columns}[c]
    \column{.5\textwidth}
        \begin{center}
        \includegraphics[height=.75\textwidth]{soulveur-screen.png}\\
        Soulver - a flat calculator
        \end{center}
    \column{.5\textwidth}
       \begin{center}
        \includegraphics[height=.75\textwidth]{m38-screen.png}\\
        m48+ Emulator
        \end{center}
    \end{columns}

\end{frame}

\subsection{Results}

\begin{frame}{Results - Traveling bill}

    \begin{block}{Instructions}
        Given various bills, calculate expenses made during a trip
    \end{block}

    \begin{itemize}
        \item Most subjects choose Skeuomorphic calculator
        \item Most subjects arrive at the correct result
        \item When asked additional question, subjects were lost
    \end{itemize}

\end{frame}

\begin{frame}{Results - Tax application}

    \begin{block}{Instructions}
        Apply additional tax (in \%) to prices, or find the tax value.
    \end{block}

    \begin{itemize}
        \item We forced subjects to use the other app.
        \item Soulver's dedicated functions were rarely used
        \item When they were used, they were confusing
        \item Subjects rarely got the correct result
        \item Subjects generally found the application unintuitive
        \item Problem when the Skeuomorphic calculator behaved differently
    \end{itemize}

\end{frame}

\begin{frame}{Results - Debriefing}
    \begin{block}{Questions}
        ``What are your thought on each app?''

        ``Which app would you prefer to use for your daily tasks?''
    \end{block}

    \begin{itemize}
        \item Preference for the Skeuomorphic calculator
        \item Did not see an use for the Flat calculator
    \end{itemize}
\end{frame}

\begin{frame}{Results - Conclusion}
    \begin{itemize}
        \item m48+ preferred when Soulver was a more complex app
        \item Soulver shows that Flat without redesign is weaker
        \item m48+ shows the limitation of Skeuomorphism
    \end{itemize}
\end{frame}


\section{Previous results}
\subsection{Flat VS Skeuomorph icons (June 2013)}

\begin{frame}{Outline}
    \tableofcontents[currentsection]
\end{frame}

\begin{frame}{Flat VS Skeuomorph icons}
	\begin{figure}
	\centering
	\includegraphics[scale=0.5]{flat.png}
	\end{figure}
    \begin{center}
      \begin{tabular}{|l|c|c|c|c|}
        \hline
        &Windows 8 & IOS7 & IOS6 & Google\\
        \hline
        Video App 	& 19\%	& 22\%	& 21\%	& 38\% \\
        Calendar App& 10\%	& 44\%	& 23\%	& 23\% \\
        Video App	& 24\%	& 22\%	& 30\%	& 24\% \\
        \hline
        Average		& 17.6\%& 29.3\%& 24.6\%& 28.3\% \\
        \hline
      \end{tabular}
    \end{center}
    	\begin{flushright}\tiny\url{www.grapheine/divers/flat-design-vs-skeumorphisme}\normalsize\end{flushright}
\end{frame}

\subsection{Windows survey (Sept 2012)}
\begin{frame}{Windows survey about new users of windows 8}

    \small
    \begin{center}
      \begin{tabular}{|l|c|c|c|c|}
        \hline
        & Windows 8 & Windows 7 & Windows XP & Others\\
        \hline
        Already used windows & 26\%	& 75\%	& 58\%	& 17\% \\
        Prefered windows	 & 25\%	& 53\%	& 20\%	& 2\% \\
        \hline
      \end{tabular}
    \end{center}
    \normalsize
    \begin{columns}[c]
      \column{.5\textwidth}
      \begin{figure}
        \centering
        \includegraphics[scale=0.25]{windows8UI.jpg}
      \end{figure}
      \column{.5\textwidth}
      \begin{figure}
        \centering
        \includegraphics[scale=0.25]{windows8Phone.png}

        \centering
        \includegraphics[scale=0.25]{windows8Tablet.png}
      \end{figure}

	\end{columns}
	\begin{flushright}\tiny\url{www.forumswindows8.com/general-discussion/windows-8-survey-half-prefer-windows-7-a-7853.htm}\normalsize\end{flushright}
\end{frame}



\subsection{Japanese study about concrete and flat design (March 2013)}
\begin{frame}{Japanese study about concrete and flat design}
	\begin{itemize}
    \item Vast choice of app, first impression is important. Design of the icon should be well choosen\\
    \begin{figure}
        \centering
        \includegraphics[scale=0.5]{UI.png}
      \end{figure}
    \item Icons analysed by 8 designers with one questionnaire, and by 42 buyers of apps with another. Tried to find best correlation between particularities to make a map
    \end{itemize}
	\begin{flushright}\tiny\url{http://design-cu.jp/iasdr2013/papers/1811-1b.pdf}\normalsize\end{flushright}
\end{frame}

\begin{frame}{Japanese study about concrete and flat design (2)}
    \begin{figure}
        \centering
        \includegraphics[scale=0.45]{caract.png}
      \end{figure}
	\begin{flushright}\tiny\url{http://design-cu.jp/iasdr2013/papers/1811-1b.pdf}\normalsize\end{flushright}
\end{frame}

\begin{frame}{Japanese study about concrete and flat design (3)}
	\begin{columns}[c]
      \column{.6\textwidth}
      \begin{figure}
        \centering
        \includegraphics[scale=0.45]{map.png}
      \end{figure}
      \column{.4\textwidth}
      \begin{itemize}
        \item Preference between 16 icons
        \item 77 participants
        \item 77\% smartphones
        \item 23\% featurephones
      \end{itemize}
	\end{columns}
	\begin{flushright}\tiny\url{http://design-cu.jp/iasdr2013/papers/1811-1b.pdf}\normalsize\end{flushright}
\end{frame}

\begin{frame}{Japanese study about concrete and flat design (4)}
	\begin{figure}
    	\centering
        \includegraphics[scale=0.6]{finalStat.png}
    \end{figure}
    \begin{itemize}
        \item 75\% prefer concrete icons
		\item Icon music is different... Old item not used anymore?
    \end{itemize}
	\begin{flushright}\tiny\url{http://design-cu.jp/iasdr2013/papers/1811-1b.pdf}\normalsize\end{flushright}
\end{frame}

\section*{Conclusion}

\begin{frame}{Conclusion}

\begin{itemize}
    \item Design trend: Skeuomorph $\rightarrow$ Flat
    \item Flat: Forced on developers, harder to design with
    \item Need to take into account design principles
\end{itemize}

\end{frame}

\begin{frame}{Thanks for your attention}
    \begin{center}
        \includegraphics[width=.5\textwidth]{end_question.png}
    \end{center}
\end{frame}


%\bibliographystyle{amsalpha}
%\bibliography{dit-paper}

\end{document}
